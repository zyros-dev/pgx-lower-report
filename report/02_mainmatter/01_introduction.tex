\chapter{Introduction}\label{sec:intro}
Databases are a heavily used type of system that rely on correctness and speed.
Nowadays, they are often the primary bottleneck in many systems - especially
on web servers and other large data applications \cite{ddia}. With modern
hardware advances, the optimal way to structure these databases has
drastically changed, but most databases are using architectures defined by older
hardware. Older databases assume the disk operations are the vast majority of
runtime, but that has shifted to the CPU for heavy queries.

Research projects typically create standalone databases. However, such approaches make distribution
harder, requiring projects to implement all supporting infrastructure themselves.
For production projects, supporting infrastructure requires implementing ACID, MVCC, query plan optimisation, and more. By using an established database, we can
address this.

\emph{pgx-lower} (PostgreSQL extension, lowering as in MLIR lowering) replaces PostgreSQL's execution engine with LingoDB's compiler
to bridge the gap of modern compilers with established systems. The name refers to a PostgreSQL extension that performs MLIR lowering. PostgreSQL's
extension system is utilised to override the executor, and shows
features that can be used within PostgreSQL to assist with this research.
One concern, however, is the additional complexities in implementation
and testing.

Chapter~\ref{ch:background} covers fundamental concepts and project definition.
Chapter~\ref{ch:related-work} provides a literature survey, followed by the solution
in Chapter~\ref{ch:project}. The result is shown and discussed in 
Chapter~\ref{ch:results}, and conclusions are drawn in Chapter~\ref{ch:conclusion}.
