\chapter{Introduction}\label{sec:intro}
Databases are a heavily used type of system that rely on correctness and speed.
Nowadays, they are often the primary bottleneck in many systems - especially
on web servers and other large data applications \cite{ddia}.

With modern hardware advances, the optimal way to structure these databases has
drastically changed, but most databases are using architectures defined by older
hardware. Older databases assume the disk operations are the vast majority of
runtime, but that has shifted to the CPU for heavy queries.

These projects typically create standalone databases, but that means that
distribution becomes harder, and the projects need to implement their own database
as well, which might require many additional steps for serious
projects. To productionise the system, this might include implementing ACID,
MVCC, query plan optimisation, and more. By using an established database, we can
address this issue.

pgx-lower replaces PostgreSQL's execution engine with LingoDB's compiler
to bridge the gap of modern compilers with established systems. PostgreSQL's
extension system is utilised to override the executor, and shows there are
features that can be used within PostgreSQL that can assist with this research.
One concern, however, is the additional complexities in implementation
and testing.

This thesis is separated into a background in Chapter~\ref{ch:background} which
includes fundamental concepts and the definition/goal of the project,
then a light literature survey will be conducted in
Chapter~\ref{ch:related-work}. The project's solution will
be introduced in Chapter~\ref{ch:project}, and finally conclusions will be
drawn in Chapter~\ref{ch:conclusion}.
